
\documentclass[journal,twoside]{IEEEtran}
%\documentclass[12pt,draftcls,onecolumn,journal]{IEEEtran}
\usepackage{amsmath}
\usepackage{amssymb}
\usepackage{cases}
\usepackage{color}
%% added by Jianmin Li%添加必须的宏包
%\newcounter{mytempeqncnt}%该宏包给跨行公式使用
\usepackage{multirow,booktabs}
\usepackage{graphicx}

\usepackage{stfloats}
\usepackage[cmyk]{xcolor}
\usepackage[tight,footnotesize]{subfigure}
\usepackage[utf8x]{inputenc}
\usepackage{color}
%
%

% If IEEEtran.cls has not been installed into the LaTeX system files,
% manually specify the path to it like:
% \documentclass[journal]{../sty/IEEEtran}

% Some very useful LaTeX packages include:
% (uncomment the ones you want to load)
% *** MISC UTILITY PACKAGES ***
%
%\usepackage{ifpdf}
% Heiko Oberdiek's ifpdf.sty is very useful if you need conditional
% compilation based on whether the output is pdf or dvi.
% usage:
% \ifpdf
%   % pdf code
% \else
%   % dvi code
% \fi
% The latest version of ifpdf.sty can be obtained from:
% http://www.ctan.org/tex-archive/macros/latex/contrib/oberdiek/
% Also, note that IEEEtran.cls V1.7 and later provides a builtin
% \ifCLASSINFOpdf conditional that works the same way.
% When switching from latex to pdflatex and vice-versa, the compiler may
% have to be run twice to clear warning/error messages.

% *** CITATION PACKAGES ***
%
%\usepackage{cite}
% cite.sty was written by Donald Arseneau
% V1.6 and later of IEEEtran pre-defines the format of the cite.sty package
% \cite{} output to follow that of IEEE. Loading the cite package will
% result in citation numbers being automatically sorted and properly
% "compressed/ranged". e.g., [1], [9], [2], [7], [5], [6] without using
% cite.sty will become [1], [2], [5]--[7], [9] using cite.sty. cite.sty's
% \cite will automatically add leading space, if needed. Use cite.sty's
% noadjust option (cite.sty V3.8 and later) if you want to turn this off
% such as if a citation ever needs to be enclosed in parenthesis.
% cite.sty is already installed on most LaTeX systems. Be sure and use
% version 5.0 (2009-03-20) and later if using hyperref.sty.
% The latest version can be obtained at:
% http://www.ctan.org/tex-archive/macros/latex/contrib/cite/
% The documentation is contained in the cite.sty file itself.

% *** GRAPHICS RELATED PACKAGES ***
%
%\ifCLASSINFOpdf
%   \usepackage[pdftex]{graphicx}
%  % declare the path(s) where your graphic files are
%  % \graphicspath{{../pdf/}{../jpeg/}}
%  % and their extensions so you won't have to specify these with
%  % every instance of \includegraphics
%  % \DeclareGraphicsExtensions{.pdf,.jpeg,.png}
%\else
%  % or other class option (dvipsone, dvipdf, if not using dvips). graphicx
%  % will default to the driver specified in the system graphics.cfg if no
%  % driver is specified.
%  % \usepackage[dvips]{graphicx}
%  % declare the path(s) where your graphic files are
%  % \graphicspath{{../eps/}}
%  % and their extensions so you won't have to specify these with
%  % every instance of \includegraphics
%  % \DeclareGraphicsExtensions{.eps}
%\fi
% graphicx was written by David Carlisle and Sebastian Rahtz. It is
% required if you want graphics, photos, etc. graphicx.sty is already
% installed on most LaTeX systems. The latest version and documentation
% can be obtained at: 
% http://www.ctan.org/tex-archive/macros/latex/required/graphics/
% Another good source of documentation is "Using Imported Graphics in
% LaTeX2e" by Keith Reckdahl which can be found at:
% http://www.ctan.org/tex-archive/info/epslatex/
%
% latex, and pdflatex in dvi mode, support graphics in encapsulated
% postscript (.eps) format. pdflatex in pdf mode supports graphics
% in .pdf, .jpeg, .png and .mps (metapost) formats. Users should ensure
% that all non-photo figures use a vector format (.eps, .pdf, .mps) and
% not a bitmapped formats (.jpeg, .png). IEEE frowns on bitmapped formats
% which can result in "jaggedy"/blurry rendering of lines and letters as
% well as large increases in file sizes.
%
% You can find documentation about the pdfTeX application at:
% http://www.tug.org/applications/pdftex

% *** MATH PACKAGES ***
%
%\usepackage[cmex10]{amsmath}
% A popular package from the American Mathematical Society that provides
% many useful and powerful commands for dealing with mathematics. If using
% it, be sure to load this package with the cmex10 option to ensure that
% only type 1 fonts will utilized at all point sizes. Without this option,
% it is possible that some math symbols, particularly those within
% footnotes, will be rendered in bitmap form which will result in a
% document that can not be IEEE Xplore compliant!
%
% Also, note that the amsmath package sets \interdisplaylinepenalty to 10000
% thus preventing page breaks from occurring within multiline equations. Use:
%\interdisplaylinepenalty=2500
% after loading amsmath to restore such page breaks as IEEEtran.cls normally
% does. amsmath.sty is already installed on most LaTeX systems. The latest
% version and documentation can be obtained at:
% http://www.ctan.org/tex-archive/macros/latex/required/amslatex/math/

% *** SPECIALIZED LMST PACKAGES ***
%
%\usepackage{algorithmic}
% algorithmic.sty was written by Peter Williams and Rogerio Brito.
% This package provides an algorithmic environment fo describing algorithms.
% You can use the algorithmic environment in-text or within a figure
% environment to provide for a floating algorithm. Do NOT use the algorithm
% floating environment provided by algorithm.sty (by the same authors) or
% algorithm2e.sty (by Christophe Fiorio) as IEEE does not use dedicated
% algorithm float types and packages that provide these will not provide
% correct IEEE style captions. The latest version and documentation of
% algorithmic.sty can be obtained at:
% http://www.ctan.org/tex-archive/macros/latex/contrib/algorithms/
% There is also a support site at:
% http://algorithms.berlios.de/index.html
% Also of interest may be the (relatively newer and more customizable)
% algorithmicx.sty package by Szasz Janos:
% http://www.ctan.org/tex-archive/macros/latex/contrib/algorithmicx/

% *** ALIGNMENT PACKAGES ***
%
%\usepackage{array}
% Frank Mittelbach's and David Carlisle's array.sty patches and improves
% the standard LaTeX2e array and tabular environments to provide better
% appearance and additional user controls. As the default LaTeX2e table
% generation code is lacking to the point of almost being broken with
% respect to the quality of the end results, all users are strongly
% advised to use an enhanced (at the very least that provided by array.sty)
% set of table tools. array.sty is already installed on most systems. The
% latest version and documentation can be obtained at:
% http://www.ctan.org/tex-archive/macros/latex/required/tools/

% IEEEtran contains the IEEEeqnarray family of commands that can be used to
% generate multiline equations as well as matrices, tables, etc., of high
% quality.

% *** SUBFIGURE PACKAGES ***
%\ifCLASSOPTIONcompsoc
%  \usepackage[caption=false,font=normalsize,labelfont=sf,textfont=sf]{subfig}
%\else
%  \usepackage[caption=false,font=footnotesize]{subfig}
%\fi
% subfig.sty, written by Steven Douglas Cochran, is the modern replacement
% for subfigure.sty, the latter of which is no longer maintained and is
% incompatible with some LaTeX packages including fixltx2e. However,
% subfig.sty requires and automatically loads Axel Sommerfeldt's caption.sty
% which will override IEEEtran.cls' handling of captions and this will result
% in non-IEEE style figure/table captions. To prevent this problem, be sure
% and invoke subfig.sty's "caption=false" package option (available since
% subfig.sty version 1.3, 2005/06/28) as this is will preserve IEEEtran.cls
% handling of captions.
% Note that the Computer Society format requires a larger sans serif font
% than the serif footnote size font used in traditional IEEE formatting
% and thus the need to invoke different subfig.sty package options depending
% on whether compsoc mode has been enabled.
%
% The latest version and documentation of subfig.sty can be obtained at:
% http://www.ctan.org/tex-archive/macros/latex/contrib/subfig/

% *** FLOAT PACKAGES ***
%
%\usepackage{fixltx2e}
% fixltx2e, the successor to the earlier fix2col.sty, was written by
% Frank Mittelbach and David Carlisle. This package corrects a few problems
% in the LaTeX2e kernel, the most notable of which is that in current
% LaTeX2e releases, the ordering of single and double column floats is not
% guaranteed to be preserved. Thus, an unpatched LaTeX2e can allow a
% single column figure to be placed prior to an earlier double column
% figure. The latest version and documentation can be found at:
% http://www.ctan.org/tex-archive/macros/latex/base/

%\usepackage{stfloats}
% stfloats.sty was written by Sigitas Tolusis. This package gives LaTeX2e
% the ability to do double column floats at the bottom of the page as well
% as the top. (e.g., "\begin{figure*}[!b]" is not normally possible in
% LaTeX2e). It also provides a command:
%\fnbelowfloat
% to enable the placement of footnotes below bottom floats (the standard
% LaTeX2e kernel puts them above bottom floats). This is an invasive package
% which rewrites many portions of the LaTeX2e float routines. It may not work
% with other packages that modify the LaTeX2e float routines. The latest
% version and documentation can be obtained at:
% http://www.ctan.org/tex-archive/macros/latex/contrib/sttools/
% Do not use the stfloats baselinefloat ability as IEEE does not allow
% \baselineskip to stretch. Authors submitting work to the IEEE should note
% that IEEE rarely uses double column equations and that authors should try
% to avoid such use. Do not be tempted to use the cuted.sty or midfloat.sty
% packages (also by Sigitas Tolusis) as IEEE does not format its papers in
% such ways.
% Do not attempt to use stfloats with fixltx2e as they are incompatible.
% Instead, use Morten Hogholm'a dblfloatfix which combines the features
% of both fixltx2e and stfloats:
%
% \usepackage{dblfloatfix}
% The latest version can be found at:
% http://www.ctan.org/tex-archive/macros/latex/contrib/dblfloatfix/

%\ifCLASSOPTIONcaptionsoff
%  \usepackage[nomarkers]{endfloat}
% \let\MYoriglatexcaption\caption
% \renewcommand{\caption}[2][\relax]{\MYoriglatexcaption[#2]{#2}}
%\fi
% endfloat.sty was written by James Darrell McCauley, Jeff Goldberg and 
% Axel Sommerfeldt. This package may be useful when used in conjunction with 
% IEEEtran.cls'  captionsoff option. Some IEEE journals/societies require that
% submissions have lists of figures/tables at the end of the paper and that
% figures/tables without any captions are placed on a page by themselves at
% the end of the document. If needed, the draftcls IEEEtran class option or
% \CLASSINPUTbaselinestretch interface can be used to increase the line
% spacing as well. Be sure and use the nomarkers option of endfloat to
% prevent endfloat from "marking" where the figures would have been placed
% in the text. The two hack lines of code above are a slight modification of
% that suggested by in the endfloat docs (section 8.4.1) to ensure that
% the full captions always appear in the list of figures/tables - even if
% the user used the short optional argument of \caption[]{}.
% IEEE papers do not typically make use of \caption[]'s optional argument,
% so this should not be an issue. A similar trick can be used to disable
% captions of packages such as subfig.sty that lack options to turn off
% the subcaptions:
% For subfig.sty:
% \let\MYorigsubfloat\subfloat
% \renewcommand{\subfloat}[2][\relax]{\MYorigsubfloat[]{#2}}
% However, the above trick will not work if both optional arguments of
% the \subfloat command are used. Furthermore, there needs to be a
% description of each subfigure *somewhere* and endfloat does not add
% subfigure captions to its list of figures. Thus, the best approach is to
% avoid the use of subfigure captions (many IEEE journals avoid them anyway)
% and instead reference/explain all the subfigures within the main caption.
% The latest version of endfloat.sty and its documentation can obtained at:
% http://www.ctan.org/tex-archive/macros/latex/contrib/endfloat/
%
% The IEEEtran \ifCLASSOPTIONcaptionsoff conditional can also be used
% later in the document, say, to conditionally put the References on a 
% page by themselves.

% *** PDF, URL AND HYPERLINK PACKAGES ***
%
%\usepackage{url}
% url.sty was written by Donald Arseneau. It provides better support for
% handling and breaking URLs. url.sty is already installed on most LaTeX
% systems. The latest version and documentation can be obtained at:
% http://www.ctan.org/tex-archive/macros/latex/contrib/url/
% Basically, \url{my_url_here}.

% *** Do not adjust lengths that control margins, column widths, etc. ***
% *** Do not use packages that alter fonts (such as pslatex).         ***
% There should be no need to do such things with IEEEtran.cls V1.6 and later.
% (Unless specifically asked to do so by the journal or conference you plan
% to submit to, of course. )

% correct bad hyphenation here
\hyphenation{op-tical net-works semi-conduc-tor}
\graphicspath{{figures/}}
\begin{document}
%
% paper title
% Titles are generally capitalized except for words such as a, an, and, as,
% at, but, by, for, in, nor, of, on, or, the, to and up, which are usually
% not capitalized unless they are the first or last word of the title.
% Linebreaks \\ can be used within to get better formatting as desired.
% Do not put math or special symbols in the title.
\title{A Fast Power Grid Frequency Estimation Approach Using Frequency-Shift Filtering}
%
%
% author names and IEEE memberships
% note positions of commas and nonbreaking spaces ( ~ ) LaTeX will not break
% a structure at a ~ so this keeps an author's name from being broken across
% two lines.
% use \thanks{} to gain access to the first footnote area
% a separate \thanks must be used for each paragraph as LaTeX2e's \thanks
% was not built to handle multiple paragraphs
%
\author{Jianmin Li,~
		Zhaosheng Teng,~Wenxuan Yao,~Yajun Wang,~Shutang You,
		and~Yilu Liu, \IEEEmembership{Fellow,~IEEE}% <-this % stops a space
		
			
		\thanks{J.  Li and Z. Teng are with the College of Electrical and Information Engineering, Hunan University, Changsha 410082, China(e-mail: ljmdzyx@163.com; tengzs@126.com)} % <-this % stops a space
		\thanks{W. Yao, Y. Wang, S. You, and Y. Liu are with the Department of Electrical Engineering and Computer Science, the University of Tennessee, Knoxville, TN, 37996, USA (e-mails:wyao3@utk.edu; ywang139@vols.utk.edu;syou3@vols.utk.edu; liu@utk.edu).}
		\thanks{W. Yao and Y. Liu are also with  Oak Ridge National Laboratory, Oak Ridge, TN, USA, 37831}
		}% <-this % stops a space

% note the % following the last \IEEEmembership and also \thanks - 
% these prevent an unwanted space from occurring between the last author name
% and the end of the author line. i.e., if you had this:
% 
% \author{....lastname \thanks{...} \thanks{...} }
%                     ^------------^------------^----Do not want these spaces!
%
% a space would be appended to the last name and could cause every name on that
% line to be shifted left slightly. This is one of those "LaTeX things". For
% instance, "\textbf{A} \textbf{B}" will typeset as "A B" not "AB". To get
% "AB" then you have to do: "\textbf{A}\textbf{B}"
% \thanks is no different in this regard, so shield the last } of each \thanks
% that ends a line with a % and do not let a space in before the next \thanks.
% Spaces after \IEEEmembership other than the last one are OK (and needed) as
% you are supposed to have spaces between the names. For what it is worth,
% this is a minor point as most people would not even notice if the said evil
% space somehow managed to creep in.

% The paper headers
\markboth{IEEE PES Letter, VOL.\ xx, NO.\ x, Sept.\ 2018} %
{LI \MakeLowercase{\textit{et al.}}:A Fast Power Grid Frequency Estimation Approach Using Frequency-Shift Filtering}
% The only time the second header will appear is for the odd numbered pages
% after the title page when using the twoside option.
% 
% *** Note that you probably will NOT want to include the author's ***
% *** name in the headers of peer review papers.                   ***
% You can use \ifCLASSOPTIONpeerreview for conditional compilation here if
% you desire.

% If you want to put a publisher's ID mark on the page you can do it like
% this:
%\IEEEpubid{0000--0000/00\$00.00~\copyright~2014 IEEE}
% Remember, if you use this you must call \IEEEpubidadjcol in the second
% column for its text to clear the IEEEpubid mark.

% use for special paper notices
%\IEEEspecialpapernotice{(Invited Paper)}

% make the title area
\maketitle

% As a general rule, do not put math, special symbols or citations
% in the abstract or keywords.
\begin{abstract}
In this letter, a fast frequency estimation approach is proposed using frequency-shift filtering. The original sampled signal is first frequency-shifted to 0 Hz nearby via multiplying a reference signal. Then a convolution average filter is applied on the shifted signal to eliminate the spectral interference caused by asynchronous sampling. Finally, frequency of power system can be estimated using the phase difference between arbitrary two points of the filtered signal. 
The approach has major advantages of low computational burden and strong anti-noise performance, which makes it appropriate for high precision and reporting rate frequency measurement in embedded systems. Comprehensive simulations are conducted to validate the accuracy and efficiency of the proposed approach.

\end{abstract}

% Note that keywords are not normally used for peerreview papers.
\begin{IEEEkeywords}
 Convolution average filter, frequency estimation, frequency-shift filtering, high reporting rate
\end{IEEEkeywords}

% For peer review papers, you can put extra information on the cover
% page as needed:
% \ifCLASSOPTIONpeerreview
% \begin{center} \bfseries EDICS Category: 3-BBND \end{center}
% \fi
%
% For peerreview papers, this IEEEtran command inserts a page break and
% creates the second title. It will be ignored for other modes.
\IEEEpeerreviewmaketitle

%----------------------------------------------------------------------
% SECTION I: Introduction
%----------------------------------------------------------------------
\section{Introduction}
% The very first letter is a 2 line initial drop letter followed
% by the rest of the first word in caps.
% 
% form to use if the first word consists of a single letter:
% \IEEEPARstart{A}{demo} file is ....
% 
% form to use if you need the single drop letter followed by
% normal text (unknown if ever used by IEEE):
% \IEEEPARstart{A}{}demo file is ....
% 
% Some journals put the first two words in caps:
% \IEEEPARstart{T}{his demo} file is ....
% 
% Here we have the typical use of a "T" for an initial drop letter
% and "HIS" in caps to complete the first word.

%BAK
%As a consequence of this, how to achieve the accuracy of frequency measurement has triggered a growing concern from both academics and businessmen.
\IEEEPARstart{F}{requency} is one the most critical parameters to indicate the operation status of  power grid. Therefore, accurate and prompt estimation of frequency is a critical task of wide-area measurement system.

Several digital signal processing   proposed  in the literature to estimate the power grid frequency such as: Discrete Fourier transform (DFT)\cite{7438889}, wavelet transform\cite{7981350}, Prony\cite{8263521}, and  Kalman filter\cite{Bagheri2016}.

Among them, DFT based methods is the most commonly
used techniques in commercial Phasor Measurement Units (PMUs) and digital relays for frequency estimation due to its advantage of easy hardware implementation and harmonic immunity\cite{7277049,5876287}. To reduce the influence of spectral leakage, Windowed Interpolation DFT (WIDFT) algorithm was used to improve the performance under the conditions of asynchronous sampling\cite{4956674}.
However, the WIDFT involve computationally expensive procedures of solving high-order equations.
A Recursive DFT (RDFT) calculation method was developed by A.~G.~Phadke and J.~S.~Thorp with significant reduction on computational complexity\cite{5519136}. To achieve the high accuracy under  off-nominal condition, calculation of least square estimation and resample is applied on RDFT\cite{chenjian}. This approach has been successfully implemented in  Frequency Disturbance
Recorders (FDRs) of FNET/Grideye with reporting rate 10 Hz for providing synchronized frequency measurement at worldwide distribution level power grid\cite{7265090}.

 \textcolor{red}{However, for  applications such as instantaneous relay protection\cite{PRC-024-1}, fast system dynamic response prediction\cite{6863288,CIGRE,68632881}, sub-synchronous resonance measurement \cite{7098450} and instant frequency-based load control \cite{7976165}, a much higher reporting rate of  measurements, e.g. 120Hz or 240 Hz, is needed to capture the transient behavior. For example, according the result in Ref.\cite{CIGRE,68632881}, a higher reporting rate leads to a faster prediction of system dynamic response under disturbance.  Unfortunately, the execution time of RDFT in real hardware implementation can not meet the real-time requirement.
 For example, the digital signal processor used in FDR needs more than 50 ms for executing a RDFT. As a result, there is no room for FDR to increase reporting rate even from 10Hz to 30Hz. }. Therefore,  the necessity to explore a more efficient frequency estimation approach with  a satisfactory performance arises.

 In this paper, a fast frequency estimation approach is proposed for practical application using frequency-shift filtering. The spectrum shifting in frequency domain  is a  digital signal processing technique used for removing interference and noise from a wanted signal\cite{fresh}. 
 The main idea of our approach is to shift fundamental components of power grid signal to 0 Hz nearby, and then use a Convolution Averaging Filter (CAF) to eliminate spectral interferences and white noise. The extensive computation of least square estimation and resample process in RDFT can be avoided. It has major advantages of high accuracy and low computational complexity.
%----------------------------------------------------------------------
% SECTION II:
%----------------------------------------------------------------------
\section{Proposed  algorithm}
Assuming  actual power grid signal distorted by harmonic is sampled at  an interval $T_s=1/f_s$ in a measurement device,  the sampled discrete-time signal $x(n)$ can be represented by
\begin{equation}
x(n)\equiv x_a(nT_s)=\sum_{h=1}^{H}A_h\cos (h2\pi f_r/f_s+\varphi_h)\label{eqn_3}
\end{equation}
where $f_r$ is the real fundamental frequency and $f_s$ is the sampling rate. $h$ stands for the order of harmonic, $A_h$ and $\varphi_h$ denote the amplitude and phase of  harmonics, respectively.
We assume the sampling frequency $f_s$ is an integral multiple of the nominal frequency $f_\mathrm{nom}$ as $f_s=Mf_\mathrm{nom}$ and define normalized angular frequency as  $\omega_r=2\pi f_r/(Mf_\mathrm{nom})$. 

According to the Euler identity, $x(n)$ can be rewritten as
\begin{equation}
x(n)=\sum_{h=1}^{H}\dfrac{A_h}{2}\left(e^{j\left(h\omega_r n+\varphi_h\right)}+e^{-j\left(h\omega_r n+\varphi_h\right)}\right)\label{eqn_4}
\end{equation}
%Hence, the discrete-time Fourier transform (DTFT) of $x(n)$ is
%\begin{align}
%X(e^{j\omega})=&\pi A_h\sum_{k=-\infty}^{\infty}\bigg[\delta\left(h\omega_r n+\varphi_h+2\pi k\right)\notag\\
%&+\left.\delta\left(h\omega_r n+\varphi_h+2\pi k\right)\right]\label{eqn_5}
%\end{align}



Defining a reference signal to shift original sampled signal, it can be expressed as 
\begin{equation}
r(n)=e^{j\omega_\mathrm{nom}n}\label{eqn_yao}
\end{equation}
where $\omega_\mathrm{nom}=2\pi /M$.
Multiplying both sides of \eqref{eqn_4} by the exponential signal $r(n)$ in time domain, we can get the shifted signal $x_s(n)$ as 

\begin{align}%=&x(n)e^{j\omega_\mathrm{nom}n}\notag\\
x_s(n)=&\sum_{h=1}^{H}\dfrac{A_h}{2}\left(e^{j\left((h\omega_r+\omega_\mathrm{nom})n+\varphi_h\right)}\right.\notag\\
&\left.+e^{-j\left((h\omega_r -\omega_\mathrm{nom})n+\varphi_h\right)}\right)\label{eqn_5}
\end{align}

\begin{figure}[t]
	\centering
	\includegraphics[width=3.5in]{freqresponse2.eps}
	\caption{Magnitude spectrum of the $p$th order CAF with $p=1,2,3,4$.}
	\label{fig_1}										
\end{figure}
\begin{figure}[t]
	\centering
	\includegraphics[width=3.5in]{frequencywide58-62.eps}
\color{red}{\caption{Frequency range test result}}
	\label{fig_2}										
\end{figure}
\begin{figure}[t]
	\centering
	\includegraphics[width=3.5in]{whitenoise2.eps}
	\caption{Noise tolerance test result}
	\label{fig_3}						
\end{figure}
\begin{figure}[t]
	\centering
	\includegraphics[width=3.5in]{har2.eps}
	\caption{Harmonic distortion test result}
	\label{fig_4}									
\end{figure}

\begin{figure}[t]
	\centering
	\includegraphics[width=3.5in]{ramp.eps}
	\color{red}{\caption{Frequency ramping test result}}
	\label{fig_5}									
\end{figure}

\begin{figure}
	\centering
	
	\subfigure[]
	{
		\includegraphics[width=1.62in]{stepchange.eps}
		\label{fig:a}
	}
	\subfigure[]
	{
		\includegraphics[width=1.62in]{anglestepchange.eps}
		\label{fig:b}
	}
 	\color{red}{\caption{Step change test results.
		(a)Frequency step change; (b)Angle  step change;}}
	\label{fig.6}
\end{figure}


It can be seen from  \eqref{eqn_5} that the frequency components of $x(n)$ are all shifted  from $h\omega_r$ and $-h\omega_r$ to $h\omega_r+\omega_\mathrm{nom}$ and $-h\omega_r+\omega_\mathrm{nom}$, respectively. Consequently, the spectral line of fundamental components are shifted around 0 Hz in frequency domain since the $f_r$ slight fluctuates around $f_\mathrm{nom}$.

To eliminate the spectral interferences of harmonic, a moving average filter is applied on the shifted signal $x_s(n)$. The  moving average filter $h_\mathrm{av}$ of length $M$ can be expressed as 
\begin{equation}
h_\mathrm{av}(m)=\begin{cases}
1/M,\qquad &m=0,1,\ldots,M-1\\
0, &\text{otherwise.}\\
\end{cases}\label{eqn_6}
\end{equation}
The frequency response of $h_\mathrm{av}(m)$ is
\begin{equation}
H(e^{j\omega})=\dfrac{1}{M}\sum_{n=0}^{M-1}e^{j\omega n}=\dfrac{\sin(\omega M/2)}{M\sin(\omega/2)}e^{-j\omega (M-1)/2}\label{eqn_7}
\end{equation}
and its magnitude frequency responses is
\begin{equation}
|H(e^{j\omega })|=\left|\dfrac{\sin(\omega M/2)}{M\sin(\omega/2)}\right|\mathrm{.}\label{eqn_8}
\end{equation}
Thus, $|H(e^{j\omega })|$ yields $0$ for the cases $\omega=2\pi k/M,k=0,1,\ldots,M-1$.

Let us divide the normalized angular frequencies into two parts after frequency shifting, that is, the negative angular frequencies $\omega_n(h)=-h\omega_r+\omega_\mathrm{nom}$ and the positive angular frequencies $\omega_p(h)=h\omega_r+\omega_\mathrm{nom}$. Since $\omega_n(1)=\omega_\mathrm{nom}-\omega_r\approx0$ for $h=1$, from \eqref{eqn_8} we have $|H(e^{j\omega_n(1)}|\approx1$. Similarity, it can be obtained that $|H(e^{j\omega_n(h)})|\approx0$ for $h\neq1$ and $|H(e^{j\omega_p(h)})|\approx0$, respectively. As as result, only the frequency component $\omega_\mathrm{nom}-\omega_r$ is reserved and the other frequency components are suppressed after applying the  filter.

To achieve better suppression effect on the interference, a $p$th order CAF is constructed as

\begin{equation}
h_p(n)=\underbrace{h_{av}(m)*\cdots*h_{av}(m)}_{p}\label{eqn_9}
\end{equation}
where $*$ denotes convolution operation, $p$ represents the number of the CAF. The  length of the $p$th order CAF is $p(M-1)+1$. Since coefficients of CAF are fixed with a certain $p$, they can be pre-computed to decrease computational burden and only one time convolution calculation between $x_s(n)$ and $h_p(n)$ is needed. 

The magnitude frequency response of the first order CAF to the fourth order CAF are illustrated in Fig.~\ref{fig_1}. It can bee observed in Fig.~\ref{fig_1} that the capacity of interference inhibition  of CAF is proportional to the convolution order $p$. Defining the length of truncation window  for original filter is $L$, the computational complexity of the proposed method is $O(pML)$, where  $O( )$ means time complexity
notation.
The signal after applying frequency shifting and CAF  can be approximated  as
\begin{equation}
x_f(n)=h_p(n)*x_s(n)\approx\sum_{n=-\infty}^{\infty}\dfrac{A_h}{2}e^{j\left((\omega_\mathrm{nom}-\omega_r)n+\varphi_1\right)}\label{eqn_10}
\end{equation}
%Fig.~\ref{fig_2} in the next page illustrates the computational procedure of the proposed algorithm.%In order to gain a better understanding of the procedure of our proposed method, we demonstrate the process graphically in

%\begin{figure*}
%	\centering
%	\includegraphics[width=7in]{Fig1.eps}
%	\caption{The computational procedure of frequency estimation with the proposed method.}
%	\label{fig_2}													
%\end{figure*}

Then the frequency deviation can obtained using phase difference of two points in $x_f(n)$ as

\begin{equation}
\Delta f=\dfrac{(\arg(x_f(n_1))-\arg(x_f(n_2))Mf_\mathrm{nom}}{2\pi(n1-n2)}.\label{eqn_12}
\end{equation}
Consequently, the estimated fundamental frequency is
\begin{equation}
f_\mathrm{est}=f_\mathrm{nom}+\Delta f\label{eqn_13}
\end{equation}
%----------------------------------------------------------------------
% SECTION III: 
%----------------------------------------------------------
\section{Simulation results and analysis}
Simulations are conducted to validate the performance of the proposed approach. The performance is also compared with FDR algorithm\cite{chenjian,7265090}, which is denoted as RDFT in this section. The nominal frequency $f_\mathrm{nom}$ is 60 Hz and the sampling rate $f_s$ is 1440 Hz. Note that both algorithms were implemented in a “sliding window” approach, i.e., continuously shifting the estimation window by a single sample, which means that the frequency estimation rate is equal to sampling rate $f_s$. Since the order $p$ influences of accuracy and computational burden, the proposed approach with $p$ from 1 to 4 are evaluated. For each test scenario, 10000 simulation runs have been performed to evaluate the statistical properties.

\textcolor{red}{For the frequency range test, the  frequency of  input sinusoidal signal  varies from 58 Hz to 62 Hz with according to IEEE Standard C37.118.1. The results is shown in Fig.~\ref{fig_2}.} It can be seen that accuracy of the proposed method is proportional to the $p$. Moreover, the proposed method with  $p>2$ can offer a better accuracy compared to the RDFT.

To evaluate the performance with white noise, Gaussian white noise with zero mean is superposed on the 59.95 Hz test signal. The accuracy is evaluated under different Signal-to-Noise Ratios (SNR) ranging from 20 to 100 dB at an increment of 10 dB. The Cramer–Rao lower bound (CRLB) is also used to provide a quantitative benchmark of measurement error as shown in Fig.~\ref{fig_3}. \textcolor{red}{In this semilog plot, it can be seen that the proposed method with larger $p$ is closer to the CRLB. The results also  demonstrates that the  proposed method with $p\geq2$ has high reliability and accuracy in a noisy signal environment than conventional RDFT, which proves robustness of the proposed method.}

For harmonic distortion test, the test signal consists of a fundamental frequency component and an odd harmonic component ranging from 3rd order  to 11th order. The magnitude is 0.1 p.u.~(20 dB with respect to the fundamental component), which is the worst condition according to the IEEE  C37.118 standard\cite{c37}. Besides, 80 dB white noise is added, which is selected based on the study of noise at the distribution level grid in Ref.~\cite{7051271}. From Fig.~\ref{fig_4}, the error of the proposed method is within 0.2 mHz and much less than the 5 mHz requirement for P-class PMU in the C37 standard. We can safely conclude that the proposed method is immune to the harmonic distortion.

\textcolor{red}{ To assess dynamic performance of the proposed method, frequency ramp and step-change tests are conducted. For the ramping test, it can be seen in Fig.~\ref{fig_5} that the frequency change trend can be well-captured using proposed method. For the step change test results in Fig.~\ref{fig.6}, the response time of the proposed is as short as 0.3 seconds when $p\le3 $, which is faster than the conventional RDFT method. From the Fig.~\ref{fig.6}(b), the frequency spike of the proposed method are much smaller than RDFT.}

\textcolor{red}{The comparative results of computation time for simulation on a 3.2 GHz Dual-core computer are listed in Table~\ref{table_1}.It is noted that in the hardware implementation, the calculation time will be longer than the simulation scenario since the speed of computation unit in an embedded system is lower than  the simulation platform.} The  simulation results confirm the superiority of the proposed algorithm in lower computational complexity. Taking account all the above results, to balance the trade-off between accuracy and estimation time, $p=2$ can be considered as reasonable selection in practical implementation  since keeping increasing $p$ does not improve the performance significantly but introducing much higher computational effort.
 
\begin{table}
	\renewcommand{\arraystretch}{1.3}
	\caption{Comparison of execution time for 10000 simulation run}
	\label{table_1}
	\centering
	\begin{tabular}{cccccc}
		\toprule
		\multirow{2}*{Method}   & \multirow{2}*{RDFT} & \multicolumn{4}{c}{Proposed method} \\
		\cmidrule{3-6}          & &$p$=1    &   $p$=2    &    $p$=3 & $p$=4    \\
		\midrule
		Time(s) & 2.57 &   0.423    &   0.444   &  0.523 & 0.796 \\
		\bottomrule
	\end{tabular}
\end{table}
%----------------------------------------------------------------------
% SECTION IV: Conclusion
%----------------------------------------------------------------------
\section{Conclusion}
In this letter, a novel  frequency estimation approach combining frequency shifting and convolution average filtering is proposed. The simulation results show that the proposed method has merit of low computational complexity, robust performance in the presence of noise, harmonic and off-nominal conditions. There advantages make it appropriate and promising for the real-time application in digital relays and  high reporting rate synchronized measurement devices.
\bibliographystyle{IEEEtran}
%\bibliography{IEEEabrv,FreqEs}
% Generated by IEEEtran.bst, version: 1.14 (2015/08/26)
\begin{thebibliography}{10}
	\providecommand{\url}[1]{#1}
	\csname url@samestyle\endcsname
	\providecommand{\newblock}{\relax}
	\providecommand{\bibinfo}[2]{#2}
	\providecommand{\BIBentrySTDinterwordspacing}{\spaceskip=0pt\relax}
	\providecommand{\BIBentryALTinterwordstretchfactor}{4}
	\providecommand{\BIBentryALTinterwordspacing}{\spaceskip=\fontdimen2\font plus
		\BIBentryALTinterwordstretchfactor\fontdimen3\font minus
		\fontdimen4\font\relax}
	\providecommand{\BIBforeignlanguage}[2]{{%
			\expandafter\ifx\csname l@#1\endcsname\relax
			\typeout{** WARNING: IEEEtran.bst: No hyphenation pattern has been}%
			\typeout{** loaded for the language `#1'. Using the pattern for}%
			\typeout{** the default language instead.}%
			\else
			\language=\csname l@#1\endcsname
			\fi
			#2}}
	\providecommand{\BIBdecl}{\relax}
	\BIBdecl
	
	\bibitem{7438889}
	L.~Zhan, Y.~Liu, and Y.~Liu, ``A clarke transformation-based dft phasor and
	frequency algorithm for wide frequency range,'' \emph{IEEE Trans. Smart
		Grid}, vol.~9, no.~1, pp. 67--77, Jan 2018.
	
	\bibitem{7981350}
	A.~Ashrafian, M.~Mirsalim, and M.~A.~S. Masoum, ``An adaptive recursive wavelet
	based algorithm for real-time measurement of power system variables during
	off-nominal frequency conditions,'' \emph{IEEE Trans. Ind. Inf.}, vol.~14,
	no.~3, pp. 818--828, March 2018.
	
	\bibitem{8263521}
	J.~Khodaparast and M.~Khederzadeh, ``Dynamic synchrophasor estimation by
	taylor-prony method in harmonic and non-harmonic conditions,'' \emph{IET
		Generation, Transmission Distribution}, vol.~11, no.~18, pp. 4406--4413,
	2017.
	
	\bibitem{Bagheri2016}
	A.~Bagheri, M.~Mardaneh, A.~Rajaei, and A.~Rahideh, ``Detection of grid voltage
	fundamental and harmonic components using kalman filter and generalized
	averaging method,'' \emph{IEEE Trans. Power Electron.}, vol.~31, no.~2, pp.
	1064--1073, Feb 2016.
	
	\bibitem{7277049}
	H.~Xue, M.~Wang, R.~Yang, and Y.~Zhang, ``Power system frequency estimation
	method in the presence of harmonics,'' \emph{IEEE Trans. Instrum. Meas.},
	vol.~65, no.~1, pp. 56--69, Jan 2016.
	
	\bibitem{5876287}
	P.~Zhang, H.~Xue, and R.~Yang, ``Shifting window average method for accurate
	frequency measurement in power systems,'' \emph{IEEE Trans. Power Delivery},
	vol.~26, no.~4, pp. 2887--2889, Oct 2011.
	
	\bibitem{4956674}
	D.~Belega and D.~Dallet, ``Multipoint interpolated dft method for frequency
	estimation,'' in \emph{2009 6th International Multi-Conference on Systems,
		Signals and Devices}, March 2009, pp. 1--6.
	
	\bibitem{5519136}
	A.~G. Phadke, J.~S. Thorp, and M.~G. Adamiak, ``A new measurement technique for
	tracking voltage phasors, local system frequency, and rate of change of
	frequency,'' \emph{IEEE Power Engineering Review}, vol. PER-3, no.~5, pp.
	23--23, May 1983.
	
	\bibitem{chenjian}
	J.~Chen, ``Accurate frequency estimation with phasor angles,'' \emph{Master.
		Dissertation, Electrical Engineering, Virginia Plytechnic Institue and State
		University, Blacksburg}, 1994.
	
	\bibitem{7265090}
	Y.~Liu, L.~Zhan, Y.~Zhang, P.~N. Markham, D.~Zhou, J.~Guo, Y.~Lei, G.~Kou,
	W.~Yao, J.~Chai, and Y.~Liu, ``Wide-area-measurement system development at
	the distribution level: An fnet/grideye example,'' \emph{IEEE Trans. Power
		Delivery}, vol.~31, no.~2, pp. 721--731, April 2016.
	
    \color{red}{\bibitem{PRC-024-1}
	\emph{PRC-024-2—Generator Frequency and Voltage Protective Relay}, North
	American Electric Reliability Corporation, May, 2015 Std.
	
	\bibitem{6863288}
	C.~Li, Y.~Liu, K.~Sun, Y.~Liu, and N.~Bhatt, ``Measurement based power system
	dynamics prediction with multivariate autoregressive model,'' in \emph{IEEE
		PES T\&D Conference and Exposition}, April 2014, pp. 1--5.
\bibitem{CIGRE}
	C.~Li, J.~Chai, Y.~Liu, N.~Bhatt, A.~D. Rosso, and E.~Farantatos, ``Power
	system dynamics prediction with measurement-based autoregressive model,''
	\emph{CIGRE Conference Grid of the Future Symposium, Houston}, 2014.

\bibitem{68632881}
	J.~Chai, ``Wide-area measurement-based applications for power system monitoring
	and dynamic modeling,'' \emph{PhD. Dissertation, Electrical Engineering, University of
		Tennessee, Knoxville,}, 2016.
	
\bibitem{7098450}
	A.~Adrees and J.~V. Milanović, ``Optimal compensation of transmission lines
	based on minimisation of the risk of subsynchronous resonance,'' \emph{IEEE
		Trans. Power Syst.}, vol.~31, no.~2, pp. 1038--1047, March 2016.
		\bibitem{7976165}
	Q.~Shi, H.~Cui, F.~Li, Y.~Liu, W.~Ju, and Y.~Sun, ``A hybrid dynamic demand
	control strategy for power system frequency regulation,'' \emph{CSEE Journal
		of Power and Energy Systems}, vol.~3, no.~2, pp. 176--185, June 2017.}
	
	\color{black}{\bibitem{fresh}
	J.~F. Adlard, ``Frequency shift filtering for cyclostationary signals,''
	\emph{PhD. Dissertation, Department of Electronics University of York}, 2000.
	
	\bibitem{c37}
	\emph{IEEE Standard for Synchrophasors for Power System C37.118.1}, Power
	System Relaying Committee of the Power Engineering Society Std.
	
	\bibitem{7051271}
	L.~Zhan, Y.~Liu, J.~Culliss, J.~Zhao, and Y.~Liu, ``Dynamic single-phase
	synchronized phase and frequency estimation at the distribution level,''
	\emph{IEEE Trans. Smart Grid}, vol.~6, no.~4, pp. 2013--2022, July 2015.}
	
\end{thebibliography}

\end{document}


