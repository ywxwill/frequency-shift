%% bare_jrnl.tex
%% V1.4a
%% 2014/09/17
%% by Michael Shell

%% see http://www.michaelshell.org/
%% for current contact information.
%%
%% This is a skeleton file demonstrating the use of IEEEtran.cls
%% (requires IEEEtran.cls version 1.8a or later) with an IEEE
%% journal paper.
%%
%% Support sites:
%% http://www.michaelshell.org/tex/ieeetran/
%% http://www.ctan.org/tex-archive/macros/latex/contrib/IEEEtran/
%% and
%% http://www.ieee.org/

%%*************************************************************************
%% Legal Notice:
%% This code is offered as-is without any warranty either expressed or
%% implied; without even the implied warranty of MERCHANTABILITY or
%% FITNESS FOR A PARTICULAR PURPOSE! 
%% User assumes all risk.
%% In no event shall IEEE or any contributor to this code be liable for
%% any damages or losses, including, but not limited to, incidental,
%% consequential, or any other damages, resulting from the use or misuse
%% of any information contained here.
%%
%% All comments are the opinions of their respective authors and are not
%% necessarily endorsed by the IEEE.
%%
%% This work is distributed under the LaTeX Project Public License (LPPL)
%% ( http://www.latex-project.org/ ) version 1.3, and may be freely used,
%% distributed and modified. A copy of the LPPL, version 1.3, is included
%% in the base LaTeX documentation of all distributions of LaTeX released
%% 2003/12/01 or later.
%% Retain all contribution notices and credits.
%% ** Modified files should be clearly indicated as such, including  **
%% ** renaming them and changing author support contact information. **
%%
%% File list of work: IEEEtran.cls, IEEEtran_HOWTO.pdf, bare_adv.tex,
%%                    bare_conf.tex, bare_jrnl.tex, bare_conf_compsoc.tex,
%%                    bare_jrnl_compsoc.tex, bare_jrnl_transmag.tex
%%*************************************************************************


% *** Authors should verify (and, if needed, correct) their LaTeX system  ***
% *** with the testflow diagnostic prior to trusting their LaTeX platform ***
% *** with production work. IEEE's font choices and paper sizes can       ***
% *** trigger bugs that do not appear when using other class files.       ***                          ***
% The testflow support page is at:
% http://www.michaelshell.org/tex/testflow/
\documentclass[journal,twoside]{IEEEtran}
%\documentclass[12pt,draftcls,onecolumn,journal]{IEEEtran}
\usepackage{amsmath}
\usepackage{amssymb}
\usepackage{cases}
\usepackage{color}
%% added by Jianmin Li%添加必须的宏包
%\newcounter{mytempeqncnt}%该宏包给跨行公式使用
\usepackage{multirow,booktabs}
\usepackage{graphicx}
\usepackage{stfloats}
\usepackage[cmyk]{xcolor}
\usepackage[tight,footnotesize]{subfigure}
%
% If IEEEtran.cls has not been installed into the LaTeX system files,
% manually specify the path to it like:
% \documentclass[journal]{../sty/IEEEtran}

% Some very useful LaTeX packages include:
% (uncomment the ones you want to load)

% *** MISC UTILITY PACKAGES ***
%
%\usepackage{ifpdf}
% Heiko Oberdiek's ifpdf.sty is very useful if you need conditional
% compilation based on whether the output is pdf or dvi.
% usage:
% \ifpdf
%   % pdf code
% \else
%   % dvi code
% \fi
% The latest version of ifpdf.sty can be obtained from:
% http://www.ctan.org/tex-archive/macros/latex/contrib/oberdiek/
% Also, note that IEEEtran.cls V1.7 and later provides a builtin
% \ifCLASSINFOpdf conditional that works the same way.
% When switching from latex to pdflatex and vice-versa, the compiler may
% have to be run twice to clear warning/error messages.

% *** CITATION PACKAGES ***
%
%\usepackage{cite}
% cite.sty was written by Donald Arseneau
% V1.6 and later of IEEEtran pre-defines the format of the cite.sty package
% \cite{} output to follow that of IEEE. Loading the cite package will
% result in citation numbers being automatically sorted and properly
% "compressed/ranged". e.g., [1], [9], [2], [7], [5], [6] without using
% cite.sty will become [1], [2], [5]--[7], [9] using cite.sty. cite.sty's
% \cite will automatically add leading space, if needed. Use cite.sty's
% noadjust option (cite.sty V3.8 and later) if you want to turn this off
% such as if a citation ever needs to be enclosed in parenthesis.
% cite.sty is already installed on most LaTeX systems. Be sure and use
% version 5.0 (2009-03-20) and later if using hyperref.sty.
% The latest version can be obtained at:
% http://www.ctan.org/tex-archive/macros/latex/contrib/cite/
% The documentation is contained in the cite.sty file itself.

% *** GRAPHICS RELATED PACKAGES ***
%
%\ifCLASSINFOpdf
%   \usepackage[pdftex]{graphicx}
%  % declare the path(s) where your graphic files are
%  % \graphicspath{{../pdf/}{../jpeg/}}
%  % and their extensions so you won't have to specify these with
%  % every instance of \includegraphics
%  % \DeclareGraphicsExtensions{.pdf,.jpeg,.png}
%\else
%  % or other class option (dvipsone, dvipdf, if not using dvips). graphicx
%  % will default to the driver specified in the system graphics.cfg if no
%  % driver is specified.
%  % \usepackage[dvips]{graphicx}
%  % declare the path(s) where your graphic files are
%  % \graphicspath{{../eps/}}
%  % and their extensions so you won't have to specify these with
%  % every instance of \includegraphics
%  % \DeclareGraphicsExtensions{.eps}
%\fi
% graphicx was written by David Carlisle and Sebastian Rahtz. It is
% required if you want graphics, photos, etc. graphicx.sty is already
% installed on most LaTeX systems. The latest version and documentation
% can be obtained at: 
% http://www.ctan.org/tex-archive/macros/latex/required/graphics/
% Another good source of documentation is "Using Imported Graphics in
% LaTeX2e" by Keith Reckdahl which can be found at:
% http://www.ctan.org/tex-archive/info/epslatex/
%
% latex, and pdflatex in dvi mode, support graphics in encapsulated
% postscript (.eps) format. pdflatex in pdf mode supports graphics
% in .pdf, .jpeg, .png and .mps (metapost) formats. Users should ensure
% that all non-photo figures use a vector format (.eps, .pdf, .mps) and
% not a bitmapped formats (.jpeg, .png). IEEE frowns on bitmapped formats
% which can result in "jaggedy"/blurry rendering of lines and letters as
% well as large increases in file sizes.
%
% You can find documentation about the pdfTeX application at:
% http://www.tug.org/applications/pdftex

% *** MATH PACKAGES ***
%
%\usepackage[cmex10]{amsmath}
% A popular package from the American Mathematical Society that provides
% many useful and powerful commands for dealing with mathematics. If using
% it, be sure to load this package with the cmex10 option to ensure that
% only type 1 fonts will utilized at all point sizes. Without this option,
% it is possible that some math symbols, particularly those within
% footnotes, will be rendered in bitmap form which will result in a
% document that can not be IEEE Xplore compliant!
%
% Also, note that the amsmath package sets \interdisplaylinepenalty to 10000
% thus preventing page breaks from occurring within multiline equations. Use:
%\interdisplaylinepenalty=2500
% after loading amsmath to restore such page breaks as IEEEtran.cls normally
% does. amsmath.sty is already installed on most LaTeX systems. The latest
% version and documentation can be obtained at:
% http://www.ctan.org/tex-archive/macros/latex/required/amslatex/math/

% *** SPECIALIZED LMST PACKAGES ***
%
%\usepackage{algorithmic}
% algorithmic.sty was written by Peter Williams and Rogerio Brito.
% This package provides an algorithmic environment fo describing algorithms.
% You can use the algorithmic environment in-text or within a figure
% environment to provide for a floating algorithm. Do NOT use the algorithm
% floating environment provided by algorithm.sty (by the same authors) or
% algorithm2e.sty (by Christophe Fiorio) as IEEE does not use dedicated
% algorithm float types and packages that provide these will not provide
% correct IEEE style captions. The latest version and documentation of
% algorithmic.sty can be obtained at:
% http://www.ctan.org/tex-archive/macros/latex/contrib/algorithms/
% There is also a support site at:
% http://algorithms.berlios.de/index.html
% Also of interest may be the (relatively newer and more customizable)
% algorithmicx.sty package by Szasz Janos:
% http://www.ctan.org/tex-archive/macros/latex/contrib/algorithmicx/

% *** ALIGNMENT PACKAGES ***
%
%\usepackage{array}
% Frank Mittelbach's and David Carlisle's array.sty patches and improves
% the standard LaTeX2e array and tabular environments to provide better
% appearance and additional user controls. As the default LaTeX2e table
% generation code is lacking to the point of almost being broken with
% respect to the quality of the end results, all users are strongly
% advised to use an enhanced (at the very least that provided by array.sty)
% set of table tools. array.sty is already installed on most systems. The
% latest version and documentation can be obtained at:
% http://www.ctan.org/tex-archive/macros/latex/required/tools/

% IEEEtran contains the IEEEeqnarray family of commands that can be used to
% generate multiline equations as well as matrices, tables, etc., of high
% quality.

% *** SUBFIGURE PACKAGES ***
%\ifCLASSOPTIONcompsoc
%  \usepackage[caption=false,font=normalsize,labelfont=sf,textfont=sf]{subfig}
%\else
%  \usepackage[caption=false,font=footnotesize]{subfig}
%\fi
% subfig.sty, written by Steven Douglas Cochran, is the modern replacement
% for subfigure.sty, the latter of which is no longer maintained and is
% incompatible with some LaTeX packages including fixltx2e. However,
% subfig.sty requires and automatically loads Axel Sommerfeldt's caption.sty
% which will override IEEEtran.cls' handling of captions and this will result
% in non-IEEE style figure/table captions. To prevent this problem, be sure
% and invoke subfig.sty's "caption=false" package option (available since
% subfig.sty version 1.3, 2005/06/28) as this is will preserve IEEEtran.cls
% handling of captions.
% Note that the Computer Society format requires a larger sans serif font
% than the serif footnote size font used in traditional IEEE formatting
% and thus the need to invoke different subfig.sty package options depending
% on whether compsoc mode has been enabled.
%
% The latest version and documentation of subfig.sty can be obtained at:
% http://www.ctan.org/tex-archive/macros/latex/contrib/subfig/

% *** FLOAT PACKAGES ***
%
%\usepackage{fixltx2e}
% fixltx2e, the successor to the earlier fix2col.sty, was written by
% Frank Mittelbach and David Carlisle. This package corrects a few problems
% in the LaTeX2e kernel, the most notable of which is that in current
% LaTeX2e releases, the ordering of single and double column floats is not
% guaranteed to be preserved. Thus, an unpatched LaTeX2e can allow a
% single column figure to be placed prior to an earlier double column
% figure. The latest version and documentation can be found at:
% http://www.ctan.org/tex-archive/macros/latex/base/

%\usepackage{stfloats}
% stfloats.sty was written by Sigitas Tolusis. This package gives LaTeX2e
% the ability to do double column floats at the bottom of the page as well
% as the top. (e.g., "\begin{figure*}[!b]" is not normally possible in
% LaTeX2e). It also provides a command:
%\fnbelowfloat
% to enable the placement of footnotes below bottom floats (the standard
% LaTeX2e kernel puts them above bottom floats). This is an invasive package
% which rewrites many portions of the LaTeX2e float routines. It may not work
% with other packages that modify the LaTeX2e float routines. The latest
% version and documentation can be obtained at:
% http://www.ctan.org/tex-archive/macros/latex/contrib/sttools/
% Do not use the stfloats baselinefloat ability as IEEE does not allow
% \baselineskip to stretch. Authors submitting work to the IEEE should note
% that IEEE rarely uses double column equations and that authors should try
% to avoid such use. Do not be tempted to use the cuted.sty or midfloat.sty
% packages (also by Sigitas Tolusis) as IEEE does not format its papers in
% such ways.
% Do not attempt to use stfloats with fixltx2e as they are incompatible.
% Instead, use Morten Hogholm'a dblfloatfix which combines the features
% of both fixltx2e and stfloats:
%
% \usepackage{dblfloatfix}
% The latest version can be found at:
% http://www.ctan.org/tex-archive/macros/latex/contrib/dblfloatfix/

%\ifCLASSOPTIONcaptionsoff
%  \usepackage[nomarkers]{endfloat}
% \let\MYoriglatexcaption\caption
% \renewcommand{\caption}[2][\relax]{\MYoriglatexcaption[#2]{#2}}
%\fi
% endfloat.sty was written by James Darrell McCauley, Jeff Goldberg and 
% Axel Sommerfeldt. This package may be useful when used in conjunction with 
% IEEEtran.cls'  captionsoff option. Some IEEE journals/societies require that
% submissions have lists of figures/tables at the end of the paper and that
% figures/tables without any captions are placed on a page by themselves at
% the end of the document. If needed, the draftcls IEEEtran class option or
% \CLASSINPUTbaselinestretch interface can be used to increase the line
% spacing as well. Be sure and use the nomarkers option of endfloat to
% prevent endfloat from "marking" where the figures would have been placed
% in the text. The two hack lines of code above are a slight modification of
% that suggested by in the endfloat docs (section 8.4.1) to ensure that
% the full captions always appear in the list of figures/tables - even if
% the user used the short optional argument of \caption[]{}.
% IEEE papers do not typically make use of \caption[]'s optional argument,
% so this should not be an issue. A similar trick can be used to disable
% captions of packages such as subfig.sty that lack options to turn off
% the subcaptions:
% For subfig.sty:
% \let\MYorigsubfloat\subfloat
% \renewcommand{\subfloat}[2][\relax]{\MYorigsubfloat[]{#2}}
% However, the above trick will not work if both optional arguments of
% the \subfloat command are used. Furthermore, there needs to be a
% description of each subfigure *somewhere* and endfloat does not add
% subfigure captions to its list of figures. Thus, the best approach is to
% avoid the use of subfigure captions (many IEEE journals avoid them anyway)
% and instead reference/explain all the subfigures within the main caption.
% The latest version of endfloat.sty and its documentation can obtained at:
% http://www.ctan.org/tex-archive/macros/latex/contrib/endfloat/
%
% The IEEEtran \ifCLASSOPTIONcaptionsoff conditional can also be used
% later in the document, say, to conditionally put the References on a 
% page by themselves.

% *** PDF, URL AND HYPERLINK PACKAGES ***
%
%\usepackage{url}
% url.sty was written by Donald Arseneau. It provides better support for
% handling and breaking URLs. url.sty is already installed on most LaTeX
% systems. The latest version and documentation can be obtained at:
% http://www.ctan.org/tex-archive/macros/latex/contrib/url/
% Basically, \url{my_url_here}.

% *** Do not adjust lengths that control margins, column widths, etc. ***
% *** Do not use packages that alter fonts (such as pslatex).         ***
% There should be no need to do such things with IEEEtran.cls V1.6 and later.
% (Unless specifically asked to do so by the journal or conference you plan
% to submit to, of course. )

% correct bad hyphenation here
\hyphenation{op-tical net-works semi-conduc-tor}
\graphicspath{{figures/}}
\begin{document}
%
% paper title
% Titles are generally capitalized except for words such as a, an, and, as,
% at, but, by, for, in, nor, of, on, or, the, to and up, which are usually
% not capitalized unless they are the first or last word of the title.
% Linebreaks \\ can be used within to get better formatting as desired.
% Do not put math or special symbols in the title.
\title{A Novel Approach for Frequency Estimation in Power System}
%
%
% author names and IEEE memberships
% note positions of commas and nonbreaking spaces ( ~ ) LaTeX will not break
% a structure at a ~ so this keeps an author's name from being broken across
% two lines.
% use \thanks{} to gain access to the first footnote area
% a separate \thanks must be used for each paragraph as LaTeX2e's \thanks
% was not built to handle multiple paragraphs
%
\author{Jianmin  Li,~
		Zhaosheng Teng,~Yilu Liu,
		and~Wenxuan Yao% <-this % stops a space
		\thanks{J.  Li and Z. Teng are with the College of Electrical and Information Engineering, Hunan University, Changsha 410082, China (e-mail: ljmdzyx@163.com; tengzs@126.com).} % <-this % stops a space
		\thanks{W. Yao and Y. Liu are with the Department of Electrical Engineering and Computer Science, the University of Tennessee, Knoxville, TN, 37996, USA (e-mails:wyao3@utk.edu, liu@utk.edu).}
		\thanks{Manuscript received March xx, 2018; revised xx xx, 2018. This work was supported by the National Natural Science Foundation of China under Grants 51377049.}}% <-this % stops a space

% note the % following the last \IEEEmembership and also \thanks - 
% these prevent an unwanted space from occurring between the last author name
% and the end of the author line. i.e., if you had this:
% 
% \author{....lastname \thanks{...} \thanks{...} }
%                     ^------------^------------^----Do not want these spaces!
%
% a space would be appended to the last name and could cause every name on that
% line to be shifted left slightly. This is one of those "LaTeX things". For
% instance, "\textbf{A} \textbf{B}" will typeset as "A B" not "AB". To get
% "AB" then you have to do: "\textbf{A}\textbf{B}"
% \thanks is no different in this regard, so shield the last } of each \thanks
% that ends a line with a % and do not let a space in before the next \thanks.
% Spaces after \IEEEmembership other than the last one are OK (and needed) as
% you are supposed to have spaces between the names. For what it is worth,
% this is a minor point as most people would not even notice if the said evil
% space somehow managed to creep in.

% The paper headers
\markboth{IEEE PES Letter} %
{LI \MakeLowercase{\textit{et al.}}: A Fast Frequency Estimation Using Frequency Shift and filter in Power System}
% The only time the second header will appear is for the odd numbered pages
% after the title page when using the twoside option.
% 
% *** Note that you probably will NOT want to include the author's ***
% *** name in the headers of peer review papers.                   ***
% You can use \ifCLASSOPTIONpeerreview for conditional compilation here if
% you desire.

% If you want to put a publisher's ID mark on the page you can do it like
% this:
%\IEEEpubid{0000--0000/00\$00.00~\copyright~2014 IEEE}
% Remember, if you use this you must call \IEEEpubidadjcol in the second
% column for its text to clear the IEEEpubid mark.

% use for special paper notices
%\IEEEspecialpapernotice{(Invited Paper)}

% make the title area
\maketitle

% As a general rule, do not put math, special symbols or citations
% in the abstract or keywords.
\begin{abstract}
In this paper, a new method is presented for the frequency estimation of power signals in power networks. It is shown that the proposed algorithm can not only accurately estimate the frequency of power grid but also require a relatively small amount of process. A frequency shifted signal is first got by shifting the original power signal with the nominal fundamental frequency. Based on the frequency shifting signal, a moving average convolution filter is constructed to filter out its irrelevant frequency components. At last, through using the phase difference between arbitrary two points of the final signal after frequency shifting and filtering, the frequency of power system can be obtained. The efficiency of the proposed method in accurate frequency estimation of power signals is demonstrated through computer simulations.
\end{abstract}

% Note that keywords are not normally used for peerreview papers.
\begin{IEEEkeywords}
Power system, discrete-time Fourier transform (DTFT), frequency estimation, frequency shifting, moving average filter.
\end{IEEEkeywords}

% For peer review papers, you can put extra information on the cover
% page as needed:
% \ifCLASSOPTIONpeerreview
% \begin{center} \bfseries EDICS Category: 3-BBND \end{center}
% \fi
%
% For peerreview papers, this IEEEtran command inserts a page break and
% creates the second title. It will be ignored for other modes.
\IEEEpeerreviewmaketitle

%----------------------------------------------------------------------
% SECTION I: Introduction
%----------------------------------------------------------------------
\section{Introduction}
% The very first letter is a 2 line initial drop letter followed
% by the rest of the first word in caps.
% 
% form to use if the first word consists of a single letter:
% \IEEEPARstart{A}{demo} file is ....
% 
% form to use if you need the single drop letter followed by
% normal text (unknown if ever used by IEEE):
% \IEEEPARstart{A}{}demo file is ....
% 
% Some journals put the first two words in caps:
% \IEEEPARstart{T}{his demo} file is ....
% 
% Here we have the typical use of a "T" for an initial drop letter
% and "HIS" in caps to complete the first word.

%BAK
%As a consequence of this, how to achieve the accuracy of frequency measurement has triggered a growing concern from both academics and businessmen.
\IEEEPARstart{F}{requency} is a primary parameter in power system, being both an essential feedback quantity to show the stable operation of the power system and one of the important indicators to evaluate the power quality. Therefore, accurate estimation of electric frequency is the basis of operation, control, and regulation of power system and electric equipment. However, with the development of power system, a lot of power electronics equipment with nonlinear characteristic are applied and many renewable energy sources are used, which in turn not only seriously pollutes public electric networks but also affects the accuracy of frequency estimation.

Many algorithms have been proposed so far in the literature to estimate the frequency of power system. Broadly, these methods could be divided into two categories: time-domain and frequency-domain methods. Time-domain analysis schemes include quasi-synchronous sampling algorithm (QSSA) \cite{Zhou2011}, Kalman filter \cite{Bagheri2016}, Prony's method \cite{Zygarlicki2010} and MUSIC algorithm \cite{Lobos2006}. Generally, the implementation of time-domain methods are usually complex. Frequency-domain analysis approaches mainly based on the fast Fourier transform (FFT) and its improved algorithm, among which the windowed interpolated FFT (WIFFT)-based algorithm is the most representative \cite{Harris1978,Wen2011,Wen2014}. Unfortunately, the WIFFT methods involve computationally expensive procedures of solving high-order equations.

The main purpose of this paper is to develop an algorithm which can not only accurately estimate the frequency but also be easy to implement. The proposed method uses the original sampling sequence of the power system in the time domain and only needs few steps. First, the original sampling signal of the power system is frequency shifted by the nominal fundamental frequency. Subsequently, in order to realize maximize preservation of the desired frequency and inhibition of the other irrelevant frequency components, a moving average convolution filter (MACF) is constructed. Finally, by calculating the phase difference between any two points of the final signal by frequency shifting and filtering, the fundamental frequency of power system can be accurately estimated.

The rest of this letter is organized as follows. The fundamental of proposed frequency estimation algorithm is detailed in section II. Section III presents the simulation results and analysis which are used to demonstrate the performance of our method in different conditions. Finally, the conclusion is drawn in Section IV.
%----------------------------------------------------------------------
% SECTION II:
%----------------------------------------------------------------------
\section{Proposed frequency estimation algorithm}
The actual power grid signal can be given by
\begin{equation}
x_a(t)=\sum_{h=1}^{H}A_h\cos(2\pi hf_rt+\varphi_h)\label{eqn_1}
\end{equation}
where $f_r$ is the real fundamental frequency of power system, $h$ stands for the order of harmonic, $A_h$ and $\varphi_h$ denote the amplitude and phase of power harmonics, respectively.

Let $f_\mathrm{nom}$ be the nominal fundamental frequency, we have
\begin{equation}
f_\mathrm{nom}=f_r+\Delta f\label{eqn_2}
\end{equation}
where $\Delta f$ is the undesired asynchronous deviation in relation to the nominal fundamental frequency.

The sampling of $x_a(t)$ with a fixed interval $T_s=1/f_s$ results in a discrete-time signal
\begin{equation}
x(n)\equiv x_a(nT_s)=\sum_{h=1}^{H}A_h\cos (h\omega_r n+\varphi_h)\label{eqn_3}
\end{equation}
where $-\infty<n<\infty$ and $\omega_r=2\pi f_r/f_s$ is the normalized real fundamental angular frequency of $x(n)$. Assume the sampling frequency is an integral multiple of the nominal fundamental frequency, that is, $f_s=Mf_\mathrm{nom}$, it follows that $\omega_r=2\pi f_r/(Mf_\mathrm{nom})$.

According to the Euler identity, $x(n)$ can be rewritten as
\begin{equation}
x(n)=\sum_{h=1}^{H}\dfrac{A_h}{2}\left(e^{j\left(h\omega_r n+\varphi_h\right)}+e^{-j\left(h\omega_r n+\varphi_h\right)}\right)\label{eqn_4}
\end{equation}
%Hence, the discrete-time Fourier transform (DTFT) of $x(n)$ is
%\begin{align}
%X(e^{j\omega})=&\pi A_h\sum_{k=-\infty}^{\infty}\bigg[\delta\left(h\omega_r n+\varphi_h+2\pi k\right)\notag\\
%&+\left.\delta\left(h\omega_r n+\varphi_h+2\pi k\right)\right]\label{eqn_5}
%\end{align}

\begin{figure}
	\centering
	\includegraphics[width=3.5in]{Fig3.eps}
	\caption{Magnitude spectrum of the $p$th order MACF with $p=1,2,3,4$.}
	\label{fig_1}													
\end{figure}
Let $\omega_\mathrm{nom}=2\pi /M$ denote the normalized nominal fundamental angular frequency of $x(n)$. By multiplying both sides of \eqref{eqn_4} by the exponential $e^{j\omega_\mathrm{nom}n}$, we have%=e^{j2\pi n/M},2\pi f_\mathrm{nom}/f_s=
\begin{align}%=&x(n)e^{j\omega_\mathrm{nom}n}\notag\\
x_s(n)=&\sum_{h=1}^{H}\dfrac{A_h}{2}\left(e^{j\left((h\omega_r+\omega_\mathrm{nom})n+\varphi_h\right)}\right.\notag\\
&\left.+e^{-j\left((h\omega_r -\omega_\mathrm{nom})n+\varphi_h\right)}\right).\label{eqn_5}
\end{align}
It can be known that the frequency components of $x(n)$ are moved from $h\omega_r$ and $-h\omega_r$ to $h\omega_r+\omega_\mathrm{nom}$ and $-h\omega_r+\omega_\mathrm{nom}$, respectively.

Then, $x_s(n)$ is undergone a MACF. But before we do that, let us first investigate the frequency characteristic of MACF. As we all know, the moving average (MA) filter of length $M$ can be expressed as 
\begin{equation}
h_\mathrm{av}(m)=\begin{cases}
1/M,\qquad &m=0,1,\ldots,M-1\\
0, &\text{otherwise.}\\
\end{cases}\label{eqn_6}
\end{equation}
Note that $M$ is also the ratio of the sampling frequency over the nominal frequency. And the discrete-time Fourier transform (DTFT) of $h_\mathrm{av}(m)$ is
\begin{equation}
H(e^{j\omega})=\dfrac{1}{M}\sum_{n=0}^{M-1}e^{j\omega n}=\dfrac{\sin(\omega M/2)}{M\sin(\omega/2)}e^{-j\omega (M-1)/2}\label{eqn_7}
\end{equation}
and its magnitude spectra is
\begin{equation}
|H(e^{j\omega })|=\left|\dfrac{\sin(\omega M/2)}{M\sin(\omega/2)}\right|\mathrm{.}\label{eqn_8}
\end{equation}
Thus, $|H(e^{j\omega })|$ yields $0$ for the cases $\omega=2\pi k/M,k=0,1,\ldots,M-1$.
%Figure \ref{fig_2} illustrates the graphs of the magnitude responses of $H(e^{j\omega})$.
%\begin{figure}
%	\centering
%	\includegraphics[width=3.5in]{Fig2.eps}
%	\caption{Magnitude responses of the MA filter.}
%	\label{fig_2}													
%\end{figure}
%into two parts after frequency shifting, that is, the negative angular frequencies $\omega_n(h)=-h\omega_r+\omega_\mathrm{nom}$ and the positive angular frequencies $\omega_p(h)=h\omega_r+\omega_\mathrm{nom}$. while $\omega_p(1)=-\omega_r+\omega_\mathrm{nom}\approx2\omega_\mathrm{nom}$, $\omega_p(1)M/2\approx2\pi$ and its  corresponding magnitude yields $|H(e^{j\omega_p(1)})|\approx0$. when $h>1$, $\omega_n(h)=-h\omega_r+\omega_\mathrm{nom}\approx(1-h)\omega_\mathrm{nom}$, $\omega_n(h)M/2\approx(1-h)\pi$ and its corresponding magnitude yields $|H(e^{j\omega_n(h)})|\approx0$ while $\omega_p(h)=-h\omega_r+\omega_\mathrm{nom}\approx(1+h)\omega_\mathrm{nom}$, $\omega_p(h)M/2\approx(1+h)\pi$ and its corresponding magnitude yields $|H(e^{j\omega_p(h)})|\approx0$. Let $\omega_h=h\omega_r+\omega_\mathrm{nom}$ be the harmonic angular frequency. 

Let us divide the normalized angular frequencies into two parts after frequency shifting, that is, the negative angular frequencies $\omega_n(h)=-h\omega_r+\omega_\mathrm{nom}$ and the positive angular frequencies $\omega_p(h)=h\omega_r+\omega_\mathrm{nom}$. Since $\omega_n(1)=\omega_\mathrm{nom}-\omega_r\approx0$ for $h=1$, from \eqref{eqn_8} we have $|H(e^{j\omega_n(1)}|\approx1$. In the same manner, it can be obtained that $|H(e^{j\omega_n(h)})|\approx0$ for $h\neq1$ and $|H(e^{j\omega_p(h)})|\approx0$, respectively. That is to say, only the frequency component $\omega_\mathrm{nom}-\omega_r$ is reserved and the other frequencies are inhibited after the MA filter.

In practice application, in order to achieve better inhibitory effect while keep the desirable frequency during the process of filtering, a $p$th order MACF is constructed as
\begin{equation}
h_p(n)=\begin{cases}
\underbrace{h_{av}(m)*\cdots*h_{av}(m)}_{p},&m=0,1,\ldots,M-1\\
0,&\text{otherwise}\\
\end{cases}\label{eqn_9}
\end{equation}
where $*$ denotes convolution operation, $p$ represents the number of the MA filters, and the length of the $p$th order MACF is $p(M-1)+1$. The magnitude spectra of the first order MACF to the fourth order MACF are illustrated in Fig.~\ref{fig_1}. With the aid of Fig.~\ref{fig_1}, we observe that the inhibition capacity of MACF is proportional to the filter order $p$ for the convolution theorem of  the Fourier Transform.
%According to the convolution theorem of Fourier Transform, we know that if two signals are convolved in the time domain, then this is equivalent to multiplying their spectra in the frequency domain. To illustrate the point, 

Finally, the signal after frequency shifting and MACF filtering can be approximated  as
\begin{equation}
x_f(n)\approx\sum_{n=-\infty}^{\infty}\dfrac{A_h}{2}e^{j\left((\omega_\mathrm{nom}-\omega_r)n+\varphi_1\right)}\label{eqn_10}
\end{equation}
%Fig.~\ref{fig_2} in the next page illustrates the computational procedure of the proposed algorithm.%In order to gain a better understanding of the procedure of our proposed method, we demonstrate the process graphically in

%\begin{figure*}
%	\centering
%	\includegraphics[width=7in]{Fig1.eps}
%	\caption{The computational procedure of frequency estimation with the proposed method.}
%	\label{fig_2}													
%\end{figure*}

Now, suppose that we select arbitrary two points $x_f(n_1)$ and $x_f(n_2)$ from $x_f(n)$. Then, it easily follows that
%we may conclude that
%Combined with the expression \eqref{eqn_2}, we also have
%\begin{equation}
%\omega_\mathrm{nom}-\omega_r=\dfrac{2\pi\Delta f}{Mf_\mathrm{nom}}
%\end{equation}
%By putting the two together, we have
\begin{equation}
\omega_\mathrm{nom}-\omega_r=\dfrac{\arg(x_f(n_1))-\arg(x_f(n_2))}{n1-n2}.\label{eqn_11}
\end{equation}
By combining \eqref{eqn_2} and \eqref{eqn_11}, we obtain the deviation frequency that
\begin{equation}
\Delta f=\dfrac{(\omega_\mathrm{nom}-\omega_r)Mf_\mathrm{nom}}{2\pi}.\label{eqn_12}
\end{equation}
Consequently, the real fundamental frequency is
\begin{equation}
f_r=f_\mathrm{nom}+\Delta f\label{eqn_13}
\end{equation}

It is worth pointing out that the duration of $x(n)$ is usually limited to $L$ samples ($0\leqslant n\leqslant L-1$) in practice. In other words, $x(n)$ is truncated by a rectangular window with the length is $L$ which will lead to a certain degree of the spectrum leakage. In spite of this, we will find that if the $L$ is long enough, the spectral leakage has little impact on our proposed algorithm.

%----------------------------------------------------------------------
% SECTION III: 
%----------------------------------------------------------------------
\section{Simulation results and analysis}
To verify the effectiveness of the proposed method, we perform simulations on signals with and without white noise using MATLAB in this section.
\subsection{Simulation without white noise}
In this simulation, the signal model given in \cite{Wen2014} and \cite{Zhang2001} is adopted. This signal is composed of eleven orders harmonics whose parameters are given in Table~\ref{table_1}. The amplitudes $A_h$ are measured in an actual electric power system and the phases $\varphi_h$ are arbitrarily chosen. The nominal frequency $f_\mathrm{nom}$ is 50Hz, the system frequency $f_r$ varies from 49.5 to 50.5 Hz with the step of 0.1 Hz, and the sampling frequency $f_s$ is 3000 Hz. Besides that, the WIFFT algorithms have the same length 1024 and adopt the 5th order fitting polynomials, the number of iterations of QSSA is 4, the sampling length of the proposed algorithm is $L$=1024 and the $4$th-order MACF is adopted.

\begin{table}
	\renewcommand{\arraystretch}{1.3}
	\caption{Parameters of the simulation signal}
	\label{table_1}
	\centering
	%\scriptsize
	\begin{tabular}{@{ }c@{ }@{ }c@{ }@{ }c@{ }@{ }c@{ }@{ }c@{ }@{ }c@{ }@{ }c@{ }@{ }c@{ }@{ }c@{ }@{ }c@{ }@{ }c@{ }@{ }c@{ }}
		\toprule
		$h$ & $1$st & $2$nd & $3$rd & $4$th & $5$th & $6$th & $7$th & $8$th & $9$th & $10$th & $11$th \\
		\midrule
		$A_h$ (V) & $240$ & $0.1$ & $12$ & $0.1$ & $2.7$ & $0.05$ & $2.1$ & $0$ & $0.3$ & $0$ & $0.6$ \\
		$\varphi_h$ ($^\circ$) & $0$ & $10$ & $20$ & $30$ & $40$ & $50$ & $60$ & - & $80$ & - & $100$ \\
		\bottomrule
	\end{tabular}
\end{table}

The absolute errors of the proposed algorithm and other reported methods, including QSSA \cite{Zhou2011}, the WIFFT algorithms based on Blackman window (BW) \cite{Harris1978}, minimize side-lobe window (MSW4(I)) \cite{Wen2011}, and the 4th-order Triangular Self-Convolution Window (TSCW) \cite{Wen2014} are listed in Table~\ref{table_2}.

From the comparative results presented in Table~\ref{table_2}, it can be seen that the proposed algorithm can effectively restrain the influence of the variation of the fundamental frequency. Its frequency estimation performance is as good as that of TSCW and is superior to the other three methods. It is worth pointing out the complexity of the new algorithm is lower than the other algorithms.
\begin{table}
	\renewcommand{\arraystretch}{1.3}
	\caption{Absolute errors of fundamental frequency by using different algorithms}
	\label{table_2}
	\centering
	%\scriptsize
	\begin{tabular}{@{ }c@{ }@{ }c@{ }@{ }c@{ }@{ }c@{ }@{ }c@{ }@{ }c@{ }}
		\toprule
		$f_r$ (Hz) & QSSA      & BW        & MSW4(I)   & TSCW       & \textbf{Our method}\\
		\midrule
		$49.5$     & $2.56E-5$ & $5.48E-6$ & $1.08E-6$ & $2.48E-10$ & $9.21E-10$ \\
		$49.6$     & $1.93E-5$ & $4.23E-6$ & $8.32E-7$ & $5.66E-10$ & $4.38E-10$ \\
		$49.7$     & $1.40E-5$ & $2.40E-6$ & $4.70E-7$ & $1.18E-10$ & $1.29E-10$ \\
		$49.8$     & $9.81E-6$ & $4.48E-7$ & $8.46E-8$ & $2.12E-9$  & $1.59E-11$ \\
		$49.9$     & $6.55E-6$ & $2.06E-6$ & $4.06E-7$ & $3.14E-9$  & $2.91E-13$ \\
		$50.0$     & $4.14E-6$ & $3.97E-7$ & $7.84E-7$ & $3.76E-9$  & $0$        \\
		$50.1$     & $2.44E-6$ & $5.16E-6$ & $1.02E-6$ & $3.40E-9$  & $6.82E-13$ \\
		$50.2$     & $1.33E-6$ & $5.54E-6$ & $1.10E-6$ & $2.69E-9$  & $5.07E-12$ \\
		$50.3$     & $6.50E-7$ & $5.01E-6$ & $1.00E-6$ & $1.99E-9$  & $4.10E-11$ \\
		$50.4$     & $2.75E-7$ & $3.54E-6$ & $7.08E-7$ & $1.42E-9$  & $3.40E-10$ \\
		$50.5$     & $9.35E-8$ & $1.15E-6$ & $2.31E-7$ & $9.90E-10$ & $2.05E-9$ \\
		\bottomrule
		\multicolumn{6}{l}{$cE-d$ denotes $c\times10^{-d}$}\\
	\end{tabular}
\end{table}
\subsection{Simulation with white noise}
Practically, the actual frequency may be corrupted with the white noise introduced by signal processing devices and transmission channels. Thus, it is beneficial to evaluate the influence of the white noise on our frequency estimation algorithm. The signal whose parameters listed in Table \ref{table_1} is superposed with zero-mean white noise. The system frequency is evaluated under different signal-to-noise ratios ranging from 20 to 100 dB at an increment of 10 dB. Meanwhile, the fundamental frequency $f_r=$50.5Hz is selected for the absolute error is bigger by using our method under non-noise condition. Furthermore, simulations are performed 1000 times for each signal-to-noise ratio (SNR) and the mean values of all the estimation results are then obtained. The mean absolute errors of fundamental frequency estimation with different methods are shown in Table~\ref{table_3}.

A close inspection of the results in Table~\ref{table_3} reveals that the absolute errors of all the algorithms are larger for lower SNR and the proposed algorithm provides the best frequency estimation performance. Specifically, because of the excellent noise reduction ability the MACF, our method improves the precision of the other algorithms 1-2 orders of magnitude in the case of low SNR.
\begin{table}
	\renewcommand{\arraystretch}{1.3}
	\caption{Absolute errors of fundamental frequency with the presence of white noise}
	\label{table_3}
	\centering
	%\scriptsize
	\begin{tabular}{@{ }c@{ }@{ }c@{ }@{ }c@{ }@{ }c@{ }@{ }c@{ }@{ }c@{ }}
		\toprule
		SNR (dB)  & QSSA       & BW        & MSW4(I)    & TSCW       & \textbf{Our method}\\
		\midrule
		$20$      & $1.18E-3$  & $8.30E-3$ & $1.00E-3$  & $1.41E-2$  & $1.51E-4$ \\
		$30$      & $6.33E-4$  & $2.65E-3$ & $3.00E-3$  & $4.65E-3$  & $4.56E-5$ \\
		$40$      & $3.54E-4$  & $8.36E-4$ & $9.78E-4$  & $1.43E-3$  & $1.45E-5$ \\
		$50$      & $6.07E-5$  & $2.57E-4$ & $3.26E-4$  & $4.58E-4$  & $4.41E-6$ \\
		$60$      & $1.54E-5$  & $7.81E-5$ & $9.60E-5$  & $1.46E-4$  & $1.36E-6$ \\
		$70$      & $1.31E-5$  & $2.58E-5$ & $3.14E-5$  & $4.67E-5$  & $4.60E-7$ \\
		$80$      & $4.17E-6$  & $8.41E-6$ & $9.82E-6$  & $1.43E-5$  & $1.49E-7$ \\
		$90$      & $1.57E-7$  & $2.72E-6$ & $3.04E-6$  & $4.74E-6$  & $4.46E-8$ \\
		$100$     & $4.14E-8$  & $1.29E-6$ & $9.66E-7$  & $1.47E-6$  & $1.38E-8$ \\
		\bottomrule
	\end{tabular}
\end{table}
%----------------------------------------------------------------------
% SECTION IV: Conclusion
%----------------------------------------------------------------------
\section{Conclusion}
In order to estimate the frequency of power system accurately, a new frequency estimation algorithm is proposed in this paper. The steps to calculate the frequency of power system by our method are very simple which only includes frequency shifting and MACF filtering. Comparing with some typical algorithms by computer simulation, it shows that the proposed frequency estimation algorithm has such advantages as easy realization, high precision and strong anti-noise performance.

\bibliographystyle{IEEEtran}
\bibliography{IEEEabrv,FreqEs}

%\begin{IEEEbiography}[{\includegraphics[width=1in,height=1.25in,clip,keepaspectratio]{JianminLi.eps}}]{Jianmin Li}
%was born in Jiangxi, China, in 1985. He received the B.Sc. and M.Sc. degrees in electrical engineering from Hunan University, Changsha, China, in 2007 and 2012, respectively, where he is currently pursuing the Ph.D. degree. 
%
%His current research interests include power system analysis, power quality measurement, and intelligent information processing.
%\end{IEEEbiography}
%
%\begin{IEEEbiography}[{\includegraphics[width=1in,height=1.25in,clip,keepaspectratio]{ZhaoshengTeng.eps}}]{Zhaosheng Teng}
%was born in Hunan, China, in 1963. He received the B.Sc., M.Sc., and Ph.D. degrees from Hunan University, Hunan, in 1984, 1995, and 1998, respectively, all in electrical engineering. 
%
%He was a Post-Doctoral Research Fellow with the National University of Defense Technology, Hunan, from 1998 to 2000. Since 2000, he has been a Professor at Hunan University. His current research interests include power quality monitoring, information fusion, and electrical measurement.
%\end{IEEEbiography}
%
%\begin{IEEEbiography}[{\includegraphics[width=1in,height=1.25in,clip,keepaspectratio]{YunpengGao.eps}}]{Yunpeng Gao (M'14)}
%was born in Liaoning, China, in 1978. He received the B.Sc., M.Sc., and Ph.D. degrees in electrical engineering from Hunan University, Changsha, China, in 2001, 2004, and 2009, respectively.
%
%Currently, he is an Associate Professor with the College of Electrical and Information Engineering, Hunan University. His interests include power distribution systems, power quality analysis, application of digital signal processing, and machine learning algorithms in power system.
%\end{IEEEbiography}
%
%
%\begin{IEEEbiography}[{\includegraphics[width=1in,height=1.25in,clip,keepaspectratio]{JunjieHe.eps}}]{Junjie He} was born in Fujian, China, in 1991. He received the B.Sc. degree in electrical engineering from Hunan University, Changsha, China, in 2010, where he is currently pursuing the M.Sc. degree.
%	
%His current research interests include power system harmonic analysis, power quality, and digital signal processing.
%\end{IEEEbiography}

\end{document}


